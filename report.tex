\documentclass[journal]{IEEEtran}

\usepackage{newtxtext,newtxmath}
\usepackage{mathrsfs}
\usepackage{graphicx}


\begin{document}

\title{Alignment of stellar halos with filaments
\\ICRAR-Pawsey Summer Studentship Report}

\author{Emily Hackett% <-this % stops a space
\thanks{Dr Chris Power and Dr Charlotte Welker - ICRAR}
\thanks{Chris Bording - Pawsey}}

% The paper headers
\markboth{Alignment of stellar halos with cosmic filaments}%
{Shell \MakeLowercase{\textit{et al.}}: ICRAR-Pawsey Summer Studentships}
% The only time the second header will appear is for the odd numbered pages
% after the title page when using the twoside option.

% make the title area
\maketitle

%%%%% ABSTRACT %%%%%
% As a general rule, do not put math, special symbols or citations
% in the abstract or keywords.
\begin{abstract}
	Previous research has shown that the physicsal structural properties of dark matter haloes can correlate with their local environment (such as filaments and voids), as those of baryonic galaxies may do also, although to a lesser extent. The significance of this correlation is that it can be seen as evidence for the theory of hierarchical structure formation in large-scale structure (as well as possibly substructure), since it demonstrates the influence of major mergers on the structural properties of galaxies. The question this project aims to address is whether this similar correlation between large-scale cosmic web structure and physical characteristics of the smaller embedded structure exists for stellar halos. The stellar halo of a galaxy is an area in which this trend may be more easily examined, since they have long orbital time periods, little dissipation effects and exist in smoother potentials in the outer regions of galaxies. Shape properties for these stellar halos can be derived similarly to those for dark matter halos, by calculating the reduced moment of inertia tensor whose eigenvalues and eigenvectors correspond to elliptic major,intermediate and minor axes. These values can then be compared to the filamentary structure around and through the halo, which is defined as its' local environment, and measured using the DisPerSe program. We find that (...) 
\end{abstract}

\IEEEpeerreviewmaketitle

%%%%% INTRO %%%%%
\section{Introduction}
\IEEEPARstart{T}{hrough} analysis of N-body hydrodynamical simulations in CDM cosmologies, clear correlations have been observed between the properties of dark matter haloes and their local environment within the cosmic web. This environment is defined by large-scale structures such as clusters, voids, filaments and sheets, along which the dark matter halo may tend to be aligned parallel or perpendicular. This alignment can be measured with respect to physical properties such as shape (via triaxiality, ellipticity and major/minor axes alignments) or angular momenta and spin. Second order correlation has also been observed between different halo mass functions, redshifts and metallicity. Some evidence has been found of a turning point mass beyond which correlation of properties to environment vanishes or is reversed \cite{hahn07b} \cite{dubois14}. 

The shape of dark matter haloes can be determined by computing the moment of inertia tensor for the halo as a whole, assuming an ellipsoidal shape (these eigenvectors and values are related to the direction and length of the principal axes of inertia) \cite{hahn07a}. Hahn et.al \cite{hahn07a} found that haloes in clusters tend to be less spherical and more prolate, whereas halos in filaments tend to be more oblate, whilst finding overall that mass in low density regions show a clearer dependence on environment, and related these results to the temperature of the surrounding flow and infall of surrounding matter, which correlated with their environment classifications (as stable manifolds). 

The fact that these halo properties correlate with the local environment suggests that the baryonic galaxies that form within them likely do so as well \cite{hahn07b}.  Bullock and Johnston \cite{bullock05} propose a way to investigate the properties of galaxies within their local environment by examining the stellar halos of these dark matter halos, since halo stars allow for testing of galaxy formation theories by having long orbital time periods, little dissipation effects and smoother potentials. These offer an insight into the question of whether structure formation is truly hierarchical on small scales, such as within the stellar halo. If so, abundant substructure would be expected \cite{bullock05}. The reason for using the stellar halo of the dark matter halo - which comprises of all baryonic matter associated with the dark matter halo - as opposed to the smaller, baryonic galaxies scattered within is that the stellar halo will hopefully be less effected by significant merger events along filaments that tend to significantly disrupt samples of galaxies.

The aim of this project was initially to look at the large-scale hydrodynamical cosmological simulation called the Horizon-AGN \cite{dubois14}, which has been previously used to show that more massive galaxies tend to be oriented perpendicular to the filament, whilst less massive are parallel. This was proposed to be because of the misalignment of galactic angular momentum during mergers, which have occurred with higher frequency in high mass galaxies. However due to time constraints and the lack of a reliable halo-finder algorithm for the huge data set, the analysis was carried out on a single cluster - the nIFTy cluster \cite{nifty}- with the intention of presenting a unified approach to later finding correlations in a large data set. The cluster is plotted for dark matter (DM) and gas densities in Figure \ref{fig:densities} as well as for temperature, using 2D slices in the xy, xz and yz planes respectively. 

%In this study, the shape of stellar haloes will be examined by making use of tidal torque theory. Linear tidal torque theory is essentially a way of calculating the angular momentum of a cosmological structure, by making use of the Zel'Dovich approximation, among others. It exists on the assumption that in the linear region of tidal torque, the Zel'dovich displacemnt can be written as separate spatial and temporal functions \cite{porciani02a}, where the spatial function is related to the tidal field tensor (deformation tensor) and the moment of inertia tensor, both of which can be calculated using available data. Re-arranging and approximating the potential as a second order Taylor expansion gives an expression for the angular momentum in terms of the deformation tensor, D (which can be calculated from the shear tidal tensor, T) and the moment of inertia tensor, I. 

%%%%%% METHODS %%%%%
\section{Methodology}
\subsection{Reduced Moment of Inertia Tensor}
The physical shape characteristics of the stellar halo can be calculated in a number of different ways, all of which influence the resulting correlations made with the background filament. An important quantity for characterising shape is the moment of inertia tensor, which can be calculated iteratively for a set of particles as per Porciani \cite{porciani02a}. A more useful version is the reduced moment of inertia tensor \cite{tenneti15} which produces a direct correlation between eigenvalues and the principle axes of the assumed elliptic halo. The reduced inertia tensor was calculated for each halo cube according to the following equation:
\begin{equation}
	\tilde{I}_{ij}=\frac{\sum_n m_n \frac{x_{ni}x_{nj}}{r^2_n}}{\sum_n m_n}, \quad \quad \text{where} \quad r^2_n=\sum_i x^2_{ni}
	\label{eq:moitensor}
\end{equation}
Where the generalised $x_i$ coordinates are given with respect to the centre of mass of the halo, which was calculated iteratively for circles of increasingly smaller radii. From this reduced inertia tensor, the eigenvectors represent the principal axes of the halo ellipsoid, and the eigenvalues correspond to the square of the principal axes' lengths.
Since the reduced moment of inertia tensor produces a real symmetric matrix, the eigenvalues and eigenvectors were calculated analytically. 
\subsection{Shape Properties of Stellar Halos}
Using the values for the major, intermediate and minor axes of the elliptic halo, and their corresponding axis vectors, the following quantities were measured. 

\IEEEPARstart{}{Sphericity}: The sphericity of the object is defined with respect to the major and minor axis (the intermediate is ignored). Plotting the sphericity of the stellar and dark matter halo as a function of radius gives an insight into the elongating effects of matter inflow along the filament, and to what extent the halos hold their shape under this force. 
\IEEEPARstart{}{Ellipticity}:
\IEEEPARstart{}{}
\subsection{Using DisPerSe to calculate filamentary structure}
The program DisPerSe \cite{sousbie11a}, standing for 'Discrete Persistent Structures Extractor' was developed by Sousbie et.al. to study the properties of filamentary structures in the cosmic web of galaxy distributions, by using the concept of persistence. The main idea is to find persistent topological structures within a data set, which can be of the form of an N-body particle set or a grid with values. These structures are identified as components of Morse-Smale complex of some input function defined over a manifold - this complex captures the relationship between the functions' gradient, its topology, and the topology of the manifold it is defined over. Further detail on this process is given in the Appendix.
The program DisPerSe outputs the skeleton structure for the given data set, which are plotted against the density background in Figure \ref{fig:density_skel}. From this skeleton structure, the unit directional vector of the corresponding filament is calculated within differing radii from the centre of mass. Each radii contains all segments from the overall skeleton that have extremities within the given sphere, as in Figure \ref{figure:density_skel} a). From here the overall direction of the filament as defined from the centre of mass is calculated by averaging the separate segments.
\subsection{Pawsey Resources}
The Horizon-AGN data set is located on the Pawsey Supercomputer Magnus in the scratch directory, and is currently being worked on by a PhD student looking to develop a halo-finder algorithm, which would allow these results to be extended to the much larger dataset. The nIFTy cluster is a much smaller dataset, and so this analysis was capable of being conducted on a personal computer. Use was made of the NECTAR resource to run the DisPerSe program, as a higher amount of memory was required to compute the Morse-Smale complex. Although no jobs were run on Magnus throughout this project, the idea was to formulate an automated process that could be applied to a large data set of many halos, which would be easily applicable to the Horizon-AGN data set. 

\begin{figure*}[!t]
\centering
	\subfloat{\includegraphics[width=2.5in]{DMwSkel.png}}
\hfil
	\subfloat{\includegraphics[width=2.5in]{GASwSkel.png}}
\caption{Gas density skeleton plotted on top of density plots for different matter}
\label{density_skel}
\end{figure*}

\section{Results}
\subsection{Sphericity and Ellipticity against Radius}
What defining features are seen in plots of sphericity and ellipticity against radius? What are the differences between stellar and DM? How does the temperature plot give any further information?
(Introduce the idea of a confounding merger and show in pictures)
\subsection{Ellipsoid plots on Density background}
What features are observed? How do these plots vary at different radii - do they show a higher level of alignment? More sphericity? 
Misalignment angle between stellar and DM components
\subsection{Skeleton plots and filament construction at small radii}
How does the filament angle change when calculated at smaller radii around the centre of mass?
How does the alignment of the stellar/DM components with the filament change for different radii?
Which is generally more aligned with the filament, the stellar or DM component?

\section{Conclusion}
Very brief - one paragraph on what was interesting, what could be investigated, how the study could be moved up to a large data set (e.g. what was the most important statistic - the alignment of stellar halo with filament).
What does this have to say about hierarchical structure formation? How much of a confounding effect can we expect major mergers to have on any large data set that this process was conducted on?


%%%%% APPENDIX %%%%%
\appendices
\section{The Morse-Smale Complex and Persistence}
Critical points are discrete sets of points where the Morse function's gradient is null, and integral lines are curves tangent to the gradient field at every point. Integral lines cover all space and their extremities are critical points, which induces a tesselation of space into regions called ascending manifolds. The set of these ascending manifolds is the Morse complex of the function. The Morse-Smale complex is an extension of this concept - the space is tesselated into regions called 'p-cells', each of which is the intersection of an ascending and descending manifold. 
Topological components of a function can be represented by pairs of positive and negative critical points called persistence pairs - where a positive critical point corresponds to a created topological component, and a negative to a destroyed topological component. The absolute difference between the value of the critical points in a pair is the persistence, which represents the lifetime of the corresponding component. DisPerSe allows the specification of a persistence threshold which allows removal of topological components with persistence lower, hence filtering noise from the Morse-Smale complex. 
% use section* for acknowledgment
\section*{Acknowledgment}
The authors would like to thank...

%%%%% REFERENCES %%%%%
\begin{thebibliography}{1}

\bibitem{lemson99}
	G.~Lemson and G.~Kauffmann, 1999, \emph{Environmental influences on dark matter haloes and consequences for the galaxies within them},\hskip 1em MNRAS 302(1) p.111-117.	
\bibitem{bailin05}
	J.~Bailin and M.~Steinmetz, 2005, \emph{Internal and external alignment of the shapes and angular momenta of LCDM halos}, \hskip 1em Astrophysics J. 627 p.647-665.
\bibitem{hahn07a}
	Hahn et.al., 2007, \emph{Properties of dark matter haloes in clusters, filaments, sheets and voids}, \hskip 1em MNRAS 375(2) p.489-499.
\bibitem{hahn07b}
	Hahn et.al., 2007, \emph{The evolution of dark matter halo properties in clusters, filaments, sheets and voids}, \hskip 1em MNRAS 381(1) p.41-51.
\bibitem{bullock05}
	J.S.~Bullock and K.V.~Johnston, 2005, \emph{Tracing galaxy formation with stellar halos. I. Methods}, \hskip 1em Astrophysical Journal Letters 635(2) p.931-949.
\bibitem{dubois14}
	Dubois et.al., 2014, \emph{Dancing in the dark: galactic properties trace spin swings along the cosmic web}, \hskip 1em MNRAS 444(2) p.1453-1468.
\bibitem{nifty}
	Frederico, S. et.al., 2016, \emph{nIFTy galaxy cluster simulations - II. Radiative models}, \hskip 1em MNRAS 459(3) p.2973-2991.
\bibitem{bowden13}
	A.~Bowden, N.W.~Evans and V.~Belokurov, 2013, \emph{Triaxial cosmological haloes and the disc of satellites}, \hskip 1em MNRAS 435(2) p.928-933.
\bibitem{rojas12}
	A.~Rojas-Nin, O.~Valenzuela, B.~Pichardo and L.A.~Aguilar, 2012, \emph{Detecting triaxiality in the galactic dark matter halo through stellar kinematics}, \hskip 1em Astrophysical Journal Letters 757(2) p.28-33.
\bibitem{tenneti15}
	A.~Tenneti, R.~Mandelbaum, T.~Di Matteo, A.~Kiessling and N.~Khandai, 2015, \emph{Galaxy shapes and alignments in the MassiveBlack-II hydrodynamic and dark matter-only simulations}, \hskip 1em MNRAS 453(1) p.469-482.
\bibitem{porciani02a}
	Porciani et.al., 2002, \emph{Testing tidal-torque theory I. Spin amplitude and direction}, \hskip 1em MNRAS 332(2) p.469-482.
\bibitem{porciani02b}
	Porciani et.al., 2002, \emph{Testing tidal-torque theory II. Alignment of inertia and shear and the characteristics of protohaloes}, \hskip 1em MNRAS 332(2) p.339-351.
\bibitem{sousbie11a}
	T.~Sousbie, 2011, \emph{The persistent cosmic web and its filamentary structure - I. Theory and implementation}, \hskip 1em MNRAS 414(1) p.350-383.
\bibitem{sousbie11b}
	T.~Sousbie, 2011, \emph{The persistent cosmic web and its filamentary structure - II. Illustrations}, \hskip 1em MNRAS 414(1) p. 384-403.
\bibitem{welker14}
	Welker et.al., 2014, \emph{Mergers drive spin swings along the cosmic web}, \hskip 1em MNRAS 445(1) p.46-50.
\bibitem{codis15}
	Codis et.al., 2015, \emph{Intrinsic alignment of simulated galaxies in the cosmic web: implications for weak lensing surveys}, \hskip 1em MNRAS 448(4) p.3391-3404.
\bibitem{chisari15}
	Chisari et.al., 2015, \emph{Intrinsic alignments of galaxies in the Horizon-AGN cosmological hydrodynamical simulation}, \hskip 1em MNRAS 454(3) p.2736-2753.

\end{thebibliography}


\begin{IEEEbiography}[{\includegraphics[width=1in,height=1.25in,clip,keepaspectratio]{ehackett}}]
	{Emily Hackett}
	Bachelor of Philosophy (Honours) student at the University of Western Australia, majoring in Physics and undergoing an Honours year in Computational Physics in 2017, and holds an Assured Pathway into Medicine which she intends to begin in 2018. Works as a volunteer for Bloom, a student organisation that acts to accelerate the growth of high-potential young entrepreneurs.
\end{IEEEbiography}



\end{document}


