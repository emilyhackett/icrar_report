Through analysis of N-body hydrodynamical simulations in CDM cosmologies, clear correlation has been observed between the properties of dark matter haloes and their local environment within the cosmic web. This environment is defined by large-scale structures such as clusters, voids, filaments and sheets, along which the dark matter halo may tend to be aligned parallel or perpendicular. This alignment can be measured with respect to physical properties such as shape (via triaxiality, oblateness, prolateness or sphericity) or angular momenta and spin. Second order correlation has also been observed between different halo mass functions, redshifts and metallicity. Some evidence has been found of a turning point mass beyond which correlation of properties to environment vanishes or is reversed \cite{hahn07b} \cite{dubois14}. 

The shape of dark matter haloes can be determined by computing the moment of inertia tensor (whose eigenvectors are related to the lengths of the principal axes of inertia) \cite{hahn07a}. Hahn et.al \cite{hahn07a} found that haloes in clusters tend to be less spherical and more prolate, whereas halos in filaments tend to be more oblate, whilst finding overall that mass in low density regions show a clearer dependence on environment, and related these results to the temperature of the surrounding flow and infall of surrounding matter, which correlated with their environment classifications (as stable manifolds). 

The fact that these halo properties correlate with the local environment suggests that the baryonic galaxies that form within them likely do so as well \cite{hahn07b}.  Bullock and Johnston \cite{bullock05} propose a way to investigate the properties of galaxies within their local environment by examining stellar halos, since halo stars allow for testing of galaxy formation theories by having long orbital time periods, little dissipation effects and smoother potentials. These offer an insight into the question of whether structure formation is truly hierarchical on small scales, such as within the stellar halo. If so, abundant substructure would be expected \cite{bullock05}. 

The particular large-scale hydrodynamical cosmological simulation used is the Horizon-AGN \cite{dubois14}, which has been previously used to show that more massive galaxies tend to be oriented perpendicular to the filament, whilst less massive are parallel. This was proposed to be because of the misalignment of galactic angular momentum during mergers, which have occurred with higher frequency in high mass galaxies. 

In this study, the shape of stellar haloes will be examined by making use of tidal torque theory. Linear tidal torque theory is essentially a way of calculating the angular momentum of a cosmological structure, by making use of the Zel'Dovich approximation, among others. It exists on the assumption that in the linear region of tidal torque, the Zel'dovich displacemnt can be written as separate spatial and temporal functions \cite{porciani02a}, where the spatial function is related to the tidal field tensor (deformation tensor) and the moment of inertia tensor, both of which can be calculated using available data. Re-arranging and approximating the potential as a second order Taylor expansion gives an expression for the angular momentum in terms of the deformation tensor, D (which can be calculated from the shear tidal tensor, T) and the moment of inertia tensor, I. 

The Horizon-AGN hydrodynamical cosmological simulation has already been used to investigate a number of different trends in galaxy spins (for example, their dependence on mergers for reorientation \cite{welker14}, or cross-correlation with the surrounding dark matter tidal field \cite{codis15}).
