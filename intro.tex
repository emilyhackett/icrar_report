\documentclass[a4paper,fleqn,usenatbib,useAMS]{mnras}

\usepackage[T1]{fontenc}
\usepackage{ae,aecompl}
\usepackage{newtxtext,newtxmath}

\title[ICRAR Report]{ICRAR Studentship Report}
\author[E. Hackett]{Emily Hackett}

\begin{document}
\label{firstpage}
\pagerange{\pageref(firstpage)--\pageref{lastpage}}
\maketitle

\section{Introduction}
Through analysis of N-body hydrodynamical simulations in CDM cosmologies, clear correlation has been observed between the properties of dark matter haloes and their local environment within the cosmic web \cite{lemson99}. This environment may be defined by large-scale structure such as filaments, voids, sheets and clusters \cite{hahn07a} \cite{bailin05}, with which the dark matter halo may be aligned to, with respect to properties such as shape and angular momentum.

The significance of such alignment is its support of the hierarchical formation thoery of galaxies, as occuring through a series of major mergers whereby the structure of a galaxy is fully determined by its merging history and local environment \cite{lemson99}.

It is uncertain, however, if such correlation occurs because galaxies result in correlated structure with their local environment because both are subject to similar merger events, or whether the local environment itself encourages such merging events.

\bibliography{report}

\end{document}


