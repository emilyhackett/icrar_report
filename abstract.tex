The properties of dark matter haloes have been shown to correlate with their local environment (such as filaments and voids), as has those of baryonic galaxies, although to a lesser extent. This correlation can be seen as evidence for the theory of hierarchical structure formation in large-scale structure as well as possibly substructure, since it demonstrates the influence of major mergers on the structural properties of galaxies. The stellar halo of a galaxy is an area in which this trend may be more easily examined, since they have long orbital time periods, little dissipation effects and exist in smoother potentials in the outer regions of galaxies. Similarly shape properties for these stellar halos can be derived, using the moment of inertia tensor, and the influence of local environment on such properties investigated.
Things that have to be decided for the project are what shape properties of the halo will be measured (e.g. calculation of the eigenvalues of the moment of inertia tensor) and also what these properties will be correlated with (the large-scale density, the classification of the local environment as a filament or void, etc.).
