\subsection{Stellar halo properties}
The physical shape characteristics of the stellar halo can be calculated in a number of different ways, all of which influence the resulting correlations made with the background filament. An important quantity for characterising shape is the moment of inertia tensor, which can be calculated iteratively for a set of particles as per Porciani\cite{porciani02a}. A more useful version is the reduced moment of inertia tensor\cite{tenneti15} which produces a direct correlation between eigenvalues and the principle axes of the assumed elliptic halo. The reduced inertia tensor was calculated for each halo cube according to the following equation:
\begin{equation}
\mathscr{I}_{ij}=m \sum_{n=1}^{N} q_{i}^{'(n)} q_{j}^{'(n)}
\label{eq:moitensor}
\end{equation}
Where the $q_{i}^{'(n)}$ are the generalised coordinates of the system with respect to the halo centre of mass, calculated by taking the average over all grid points.
